\exercice
Considereu el trinomi de segon grau $f: x\mapsto (( a|facteur("X") )) (( b|facteur("*sox") )) (( c|facteur("so") ))$.

\begin{enumerate}
\item
\begin{enumerate}
    \item Demostreu que per a qualsevol $x\in\mathbb{R}$, es té : $f\,(x)=(( a|facteur )) \,\left( x (( -x1|facteur("so") )) \right) \, \left( x (( -x2|facteur("so") )) \right) $.
    \item Demostreu que per a qualsevol $x\in\mathbb{R}$, es té : $f\,(x)=(( a|facteur )) \,\left( x (( -alpha|facteur("so") )) \right)^2 (( beta|facteur("so") ))$.
\end{enumerate}
\item Resoleu les equacions següents triant la forma apropiada de $f$.
\begin{enumerate}
\item $f\,(x)=0$
\item $f\,(x)=(( c|facteur ))$
\item $f\,(x)=(( beta|facteur ))$
\end{enumerate}
\item
\begin{enumerate}
\item Completeu la taula de variacions de $f$.
\item Completeu la taula de signes de $f$.
\end{enumerate}
\item Contesteu les següents preguntes utilitzant la taula de signes i variacions.
\begin{enumerate}
\item Resoleu $f(x)\geqslant0$.
\item Quin és l'extrem de $f$ ? És un màxim o un mínim ?  En quin valor de $x$ s'aconsegueix ?
\end{enumerate}
\end{enumerate}
