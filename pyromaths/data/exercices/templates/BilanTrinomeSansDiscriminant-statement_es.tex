\exercice
Considerar el trinomio de segundo grado $f: x\mapsto (( a|facteur("X") )) (( b|facteur("*sox") )) (( c|facteur("so") ))$.

\begin{enumerate}
\item
\begin{enumerate}
    \item Mostrar que para todo $x\in\mathbb{R}$, se tiene : $f\,(x)=(( a|facteur )) \,\left( x (( -x1|facteur("so") )) \right) \, \left( x (( -x2|facteur("so") )) \right) $.
    \item Mostrar que para todo $x\in\mathbb{R}$, se tiene : $f\,(x)=(( a|facteur )) \,\left( x (( -alpha|facteur("so") )) \right)^2 (( beta|facteur("so") ))$.
\end{enumerate}
\item Resolver las ecuaciones siguientes eligiendo la forma apropiada de $f$.
\begin{enumerate}
\item $f\,(x)=0$
\item $f\,(x)=(( c|facteur ))$
\item $f\,(x)=(( beta|facteur ))$
\end{enumerate}
\item
\begin{enumerate}
\item Completar la tabla de variaciones de $f$.
\item Completar la tabla de signos de $f$.
\end{enumerate}
\item Responder a las siguientes cuestiones usando la tabla de signos y variaciones.
\begin{enumerate}
\item Résoudre $f(x)\geqslant0$.
\item ¿ Cual es el extremo de $f$ ? ¿ Es un máximo o un mínimo ? ¿ En que valor de $x$ se alcanza ?
\end{enumerate}
\end{enumerate}
