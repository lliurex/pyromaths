\exercice*
\begin{enumerate}

  \item
    \begin{enumerate}
      \item
        \begin{itemize}
          \item Puesto que $A\,( (( X[0]|facteur ))~;~(( Y[0]|facteur )) )$ está en la curva de $f$, entonces $f\,( (( X[0]|facteur )) )=(( Y[0]|facteur ))$, soit
            $a \times (( X[0]|facteur ))^2+b\times (( X[0]|facteur ))+c=(( Y[0]|facteur ))$,
            es decir $(( (X[0]**2)|facteur(variable="a") )) + (( X[0]|facteur(variable="b") )) + c = (( Y[0]|facteur ))$.
          \item Igualmente, puesto que $B\,( (( X[1]|facteur ))~;~(( Y[1]|facteur )) )$ está en la curva de $f$, entonces $f\,( (( X[1]|facteur )) )=(( Y[1]|facteur ))$, si
            $a \times (( X[1]|facteur ))^2+b\times (( X[1]|facteur ))+c=(( Y[1]|facteur ))$,
            es decir $(( (X[1]**2)|facteur(variable="a") )) + (( X[1]|facteur(variable="b") )) + c = (( Y[1]|facteur ))$.
          \item Finalmente, puesto que $C\,( (( X[2]|facteur ))~;~(( Y[2]|facteur )) )$ está en la curva de $f$, entonces $f\,( (( X[2]|facteur )) )=(( Y[2]|facteur ))$, si
            $a \times (( X[2]|facteur ))^2+b\times (( X[2]|facteur ))+c=(( Y[2]|facteur ))$,
            es decir $(( (X[2]**2)|facteur(variable="a") )) + (( X[2]|facteur(variable="b") )) + c = (( Y[2]|facteur ))$.
        \end{itemize}

        De donde se decuce el siguiente sistema :
        \[ \left\{\begin{array}{rcl}
            (( (X[0]**2)|facteur(variable="a") )) + (( X[0]|facteur(variable="b") )) + c &=& (( Y[0]|facteur )) \\
            (( (X[1]**2)|facteur(variable="a") )) + (( X[1]|facteur(variable="b") )) + c &=& (( Y[1]|facteur )) \\
            (( (X[2]**2)|facteur(variable="a") )) + (( X[2]|facteur(variable="b") )) + c &=& (( Y[2]|facteur )) \\
        \end{array}\right.\]
      \item

\begin{align*}
        \left\{\begin{array}{rcl}
            (( (X[0]**2)|facteur(variable="a") )) + (( X[0]|facteur(variable="b") )) + c &=& (( Y[0]|facteur )) \\
            (( (X[1]**2)|facteur(variable="a") )) + (( X[1]|facteur(variable="b") )) + c &=& (( Y[1]|facteur )) \\
            (( (X[2]**2)|facteur(variable="a") )) + (( X[2]|facteur(variable="b") )) + c &=& (( Y[2]|facteur )) \\
        \end{array}\right.
&\iff
\begin{pmatrix}
(( (X[0]**2)|facteur(variable="a") )) + (( X[0]|facteur(variable="b") )) + c \\
(( (X[1]**2)|facteur(variable="a") )) + (( X[1]|facteur(variable="b") )) + c \\
(( (X[2]**2)|facteur(variable="a") )) + (( X[2]|facteur(variable="b") )) + c \\
\end{pmatrix} = (( Y|zip|matrice ))\\
&\iff
(( M|matrice)) \times (( [["a"], ["b"], ["c"]]|matrice )) = (( Y|zip|matrice )) \\
&\iff M\, X=R
\end{align*}

        Con : $M= (( M|matrice ))$, $X= (( [["a"], ["b"], ["c"]]|matrice ))$ et $R= (( Y|zip|matrice ))$.
    \end{enumerate}
  \item
    Puest que $M$ es inversible, y que $M\,X = R$, entonces $X = M^{-1}\times R$.

    En la calculadora,obtenemos
    $M^{-1}\times R=(( M|matrice ))^{-1}\times ((Y|zip|matrice)) = (( A|zip|matrice ))$.

    Así, $a=(( A[0]|facteur ))$, $b=(( A[1]|facteur ))$, y $c=(( A[2]|facteur ))$.
  \item
Usando los valores de $a$, $b$, y $c$ calculados anteriormente, conocemos la expresión de la función : $f\,(x) = (( A[0]|facteur("X") ))  (( A[1]|facteur("sox*"))) (( A[2]|facteur("so") ))$.

Ahora podemos calcular la imagen de $(( x|facteur ))$ con esta función :
\begin{align*}
f\,( (( x|facteur )) )
&= (( A[0]|facteur )) \times (( x|facteur ))^2 (( A[1]|facteur("so") )) \times (( x|facteur)) (( A[2]|facteur("so") ))\\
&= (( (A[0]*x**2+A[1]*x+A[2])|facteur ))
\end{align*}
Entonces $f\,( (( x|facteur )) ) = (( (A[0]*x**2+A[1]*x+A[2])|facteur ))$.
\end{enumerate}

