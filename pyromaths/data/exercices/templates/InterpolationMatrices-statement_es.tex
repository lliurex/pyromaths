\exercice
Dentro de una base ortonormal, determinar la ecuación de una función cuya curva pasa por los puntos.
$A\,( (( X[0]|facteur )) ~;~ (( Y[0]|facteur )) )$,
$B\,( (( X[1]|facteur )) ~;~ (( Y[1]|facteur )) )$ et
$C\,( (( X[2]|facteur )) ~;~ (( Y[2]|facteur )) )$.

Buscar un trinomio de segundo grado, es decir una función $f$ definida en $\interval[open]{-\infty}{+\infty}$ por
 \mbox{$f\,(x) = a\,x^2 + b\,x + c$} o $a$, $b$ y $c$ son tres números reales, que estamos tratando de determinar.

  \begin{enumerate}
    \item 
      \begin{enumerate}
        \item A partir de los datos del enunciado, escribir un sistema de ecuaciones que exprese esta situación.
        \item Sabiendo que el sistema precendente es equivalente a : $M\,X = R$ con

          $M = (( M|matrice ))$, $X= (( [["a"], ["b"], ["c"]]|matrice ))$ y $R$ una matriz de columnas.
      \end{enumerate}
  \end{enumerate}
  \begin{enumerate}
      \setcounter{enumi}{1}
    \item Teniendo en cuenta que la matriz $M$ es invertible.
      Determinar los valores de los coeficientes $a$, $b$ y $c$, detallando los cálculos.
    \item ¿ Cuál es el valor de $f\,( (( x|facteur )) )$ ?

  \end{enumerate}
